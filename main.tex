\documentclass[12pts, letterpaper]{article}
\usepackage[utf8]{inputenc}

\title{Asistent UAIC}
\author{Malos Mihai Gabriel }
\date{April 2018}

\begin{document}

\maketitle
    Aplicatia “Asistent UAIC” vine in ajutorul studentilor, pentru a le usura atat organizarea si prioritizarea obiectelor de studiu, precum si interactiunea cu grupul din care fac parte.\\

    Aplicatia poate fi separate in doua parti: partea de “retea de socializare” si partea de asistent personal.
	Prima parte are ca scop  facilitarea comunicarii dintre student, in cadrul aceluasi grup, acelorasi materii, precum si obtinerea de resurse legate de anumite materii de studiu.	\\
	
	
    La prima logare, studentul va trebui sa introduca numele,prenumele, anul si grupa din care face parte. Dupa acest pas, user-ul va fi repartizat grupei din care face parte si va fi introdus in chat-ul grupei respective. In plus, va fi repartizat si in grupele aferente materiilor de studiu (optionalele sunt incluse). Apoi, in prima instanta i se va construi orarul initial. Utilizatorul va fi notificat in legatura cu orice modificare din orar, atat timp cat va fi conectat la internet.\\


	Grupele pe materii de studiu nu vor  fi  formate din grupele individuale, ci mai degraba toate grupele din semian. In comparative cu grupele individuale, aceste grupe nu vor avea un chat de grupa si vor fi folosite doar pentru a distribui materiale intre student si pentru a comunica anunturi ce tin de materia respectiva (recuperari,prezentari etc. ).In ceea ce priveste grupele individuale, dupa cum am mentionat mai sus, user-ul va avea la dispozitie un chat pentru toata grupa. In plus, se pot posta stiri/anunturi de catre fiecare student, se pot organiza meeting-uri de grupa, care vor fi administrate de un admin ales de grup.
	
	
	In cazul in care se organizeaza o intalnire de grup, aplicatia va permite setarea unui reminder pentru intalnirea respectiva. Acest lucru se va intampla doar daca, dupa momentul stabilirii intalnirii, utilizatorul a accesat macar o data aplicatia (s-a putut conecta la server).
	
	
	A doua parte a aplicatiei, cea de assistent personal, este reprezentata de o aplicatie android care a fi instalata pe telefonului utilizatorului. Aceasta aplicatie va pune la dispozitia studentului orarul curent, care se va updata periodic(sau in cazul in care apare o modificare). Pe langa aceasta, orarul va putea fi modificat, astfel incat studentul sa poata participa si la orele altei grupe, schimbarea aceasta reflectandu-se in orar. Pe langa aceasta in cazul in care studentul are un loc de munca part-time, aplicatia se va oferi sa construiasca un orar in concordanta cu orele de lucru pe care studentul trebuie sa le lucreze pe saptamana, dar si orarul de la facultate.In acest caz se va putea “renunta” la anumite cursuri/seminarii.

\end{document}
